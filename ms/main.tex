\documentclass[12pt]{extarticle}
\usepackage{geometry}
\geometry{
a4paper,
total={170mm,257mm},
left=20mm,
top=20mm,
headheight=12pt
}

\usepackage[parfill]{parskip} % Activate to begin paragraphs with an empty line rather than an indent
\usepackage{graphicx} % Use pdf, png, jpg, or eps§ with pdflatex; use eps in DVI mode
% TeX will automatically convert eps --> pdf in pdflatex
\graphicspath{ {./Figures/} }
\usepackage{subcaption}
\usepackage{float}

\usepackage{amssymb,amsmath,amsthm}
\usepackage{commath}
\usepackage[hyphens]{url}
\usepackage[dvipsnames]{xcolor}
\usepackage[unicode=true,colorlinks=true,urlcolor=CadetBlue,citecolor=black,linkcolor=black]{hyperref}
\PassOptionsToPackage{hyphens}{url} % url is loaded by hyperref
\usepackage{authblk}
\usepackage{longtable}
\usepackage{multirow}
\usepackage{booktabs}
\usepackage{lipsum}  
\usepackage[title,page]{appendix}
\usepackage{chngcntr}
      
%SetFonts
% newtxtext+newtxmath
\usepackage{newtxtext} %loads helv for ss, txtt for tt
\usepackage{amsmath}
\usepackage[bigdelims]{newtxmath}
\usepackage[T1]{fontenc}
\usepackage{textcomp}
%SetFonts

% less space before sections 
% \@startsection {NAME}{LEVEL}{INDENT}{BEFORESKIP}{AFTERSKIP}{STYLE} 
%            optional * [ALTHEADING]{HEADING} 
\makeatletter
 \renewcommand\section{\@startsection {section}{1}{\z@}%
     {-2.5ex \@plus -1ex \@minus -.2ex}%
     {1.3ex \@plus.2ex}%
    {\Large\bfseries}}
    
% Species names
%% Meta-Command for defining new species macros
\usepackage{xspace}

\newcommand{\species}[3]{%
  \newcommand{#1}{\gdef#1{\textit{#3}\xspace}\textit{#2}\xspace}}
  
\species{\yeast}{Saccharomyces cerevisiae}{S.~cerevisiae}
\species{\calbicans}{Candida albicans}{C.~albicans}
\species{\cneoformans}{Cryptococcus neoformans}{C.~neoformans}

% line numbers
 \usepackage[displaymath, mathlines]{lineno}
 \renewcommand\linenumberfont{\normalfont\small\sffamily}
\linenumbers
\modulolinenumbers[2]

% Yoav & Lee commands
\newcommand*{\tr}{^\intercal}
\let\vec\mathbf
\newcommand{\matrx}[1]{{\left[ \stackrel{}{#1}\right]}}
\newcommand{\diag}[1]{\mbox{diag}\matrx{#1}}
\newcommand{\goesto}{\rightarrow}
\newcommand{\dspfrac}[2]{\frac{\displaystyle #1}{\displaystyle #2} }
\newtheorem{theorem}{Theorem}
\newtheorem{corollary}{Corollary}
\newtheorem{lemma}{Lemma}
\newtheorem{remark}{Remark}
\newtheorem{result}{Result}
\renewcommand\qedsymbol{} % no square at end of proof
\newcommand{\cl}{\mathbf{L}}
\newcommand{\cj}{\mathbf{J}}
\newcommand{\ci}{I}

% Supplementary
% https://support.authorea.com/en-us/article/how-to-create-an-appendix-section-or-supplementary-information-1g25i5a/
\newcommand{\beginsupplement}{%
      	\setcounter{table}{0}
        \renewcommand{\thetable}{S\arabic{table}}%
        \setcounter{figure}{0}
        \renewcommand{\thefigure}{S\arabic{figure}}%
		\setcounter{equation}{0}
        \renewcommand{\theequation}{A\arabic{equation}}%
}

% NatBib
\usepackage[round,colon]{natbib}

% Title page
\title{Cultural Transmission Can Explain the Evolution of Cooperation}

% Authors
\renewcommand\Affilfont{\small}

\author[1]{Dor Cohen}
\author[2]{Ohad Lewin-Epstein}
\author[3]{Marcus W. Feldman}
\author[a,*]{Yoav Ram}

\affil[1]{School of Computer Science, Interdisciplinary Center Herzliya, Herzliya, Israel}
\affil[2]{School of Plant Sciences and Food Security, Tel Aviv University, Tel Aviv, Israel}
\affil[3]{Department of Biology, Stanford University, Stanford, CA}
\affil[*]{Corresponding author: yoav@yoavram.com}

\date{\today}

\begin{document}
\maketitle

% Abstract
%\begin{abstract}
%\end{abstract}

\pagebreak


%%%%%%%%%%%%%%%%%%%%%%%%%%%
%%% Introduction
\section*{Introduction}
Cooperative behavior can harm an individual's fitness and increase the fitness of its conspecifics or competitors~\citep{axelrod1981evolution}.
Nevertheless, cooperative behavior appears to occur in many non-human animals~\citep{dugatkin1997cooperation}, for example rats~\citep{rice1962altruism} and birds~\citep{krams2008experimental}.
Evolution of cooperative behavior remains an important conundrum in evolutionary biology.

\emph{Kin selection} theory posits that natural selection can favor cooperation between related individuals. The importance of relatedness to the evolution of cooperation and altruism was shown by~\citet{hamilton1964genetical}. According to Hamilton, for an allele that determines cooperative behavior to increase in frequency, the reproductive cost to the actor that cooperates, $c$, must be less than the benefit to the recipient, $b$, times the `relatedness' between the recipient and the actor, $r$. This `relatedness' coefficient $r$ measures the correlation between the gene in the actor and the gene in the recipient.
This condition is also known as Hamilton's rule:
\begin{equation} \label{eq:hamilton_rule}
c < b \cdot r.
\end{equation}

\citet{Eshel1982} have studied a relevant model for the evolution of cooperative behavior under vertical transmission. Their model included \emph{assortative meeting}, or non-random encounters. That is, if a fraction $m$ of the population interacts with an individual of the same phenotype, and $1-m$ interacts randomly. 
Such assortative meeting may be due, for example, to population structure or active partner choice.
In their model, cooperative behavior can evolve if \footnote{In an extended model, which allows an individual to encounter $N$ individuals before choosing a partner, the righthand side is multiplied by $E[N]$, the expected number of encounters \citep[eq.~4.6]{Eshel1982}.
}. 
\citep[eq.~3.2]{Eshel1982}
\begin{equation} \label{eq:assortative_meeting}
c < b \cdot m,
\end{equation}
where $b$ and $c$ are the benefit and cost of cooperation
Here, $m$ takes the role of the relatedness $r$.


These theories assume that cooperation is genetically determined, which raises the question: \emph{Is it possible that cooperation is determined by non-genetic factors?}
Culture has significant impact on the behavior of humans~\citep{ihara2004cultural,jeong2018bronze} as well as non-human animals~\citep{bonner2018evolution}.
Here we attempt to determine to what extent the evolution of cooperative behavior can be explained by \emph{cultural transmission},
which allows an individual to acquire attitudes and behavioral traits from other individuals in its social group through imitation, learning, or other modes of communication \citep{cavalli1981cultural,richerson2008not}.
\citet{feldman1985gene} introduced the first model for the evolution of altruism by cultural transmission.
They showed that under vertical (parent-to-offspring) cultural transmission, Hamilton's rule does not govern the evolution of parent-to-offspring or sib-to-sib altruism.

\emph{Non-vertical transmission} may be either horizontal or oblique: horizontal transmission occurs between individuals from the same generation, while oblique transmission occurs from adults to unrelated offspring. 
Evolution under either of these transmission models  can be be more rapid than under pure vertical transmission~\citep{cavalli1981cultural,ram2018evolution}.
\citet{lewin2017microbes} have demonstrated that non-vertical transmission, mediated by microbes that manipulate their host behavior, can help to explain the evolution of cooperative behavior. Interestingly, some of their analysis can be applied to cultural transmission, because models of cultural transmission are mathematically similar to those for transmission of infectious diseases~\citep{cavalli1981cultural}.

Here we hypothesize that non-vertical cultural transmission can explain the evolution of cooperation. 
To test this hypothesis, we suggest a model in which behavioral changes are mediated by cultural transmission that can occur during social interactions. For example, if an individual interacts with a cooperative individual, it might learn that cooperation is a positive behavior and will be cooperative in the future. 
We develop cultural evolution models that include both vertical and non-vertical transmission of cooperation and investigate these models using mathematical analysis and simulations.  
Our results demonstrate cultural transmission can facilitate the evolution of cooperation even when genetic transmission cannot.
These results suggest that further research on the evolution of cooperation should account for non-vertical transmission and that treatment of cooperation as a cultural, rather then genetic trait, can lead to a better understanding of this important and enigmatic phenomenon.


%%%%%%%%%%%%%%%%%%%%%%%%%%%
% Models
\section*{Models}

We focus on the evolution of cooperation in a fully mixed population where cooperation is modeled using the \emph{prisoner's dilemma}. % yr: add ref

Consider a very large population whose members are characterized by their phenotype $\phi$, which can be of two types, $\phi=A$ for cooperators or $\phi=B$ for defectors.
An offspring inherits its phenotype from its parent via vertical transmission with probability $v$ or from a random individual in the parental population via oblique transmission with probability $(1-v)$. 
Following~\citet{ram2018evolution}, given that the parent phenotype is $\phi$ and assuming uni-parental inheritance, % yr: cite Zefferman Behav Ecol 2016
the conditional probability that the phenotype $\phi'$ of the offspring is $A$ is 

\begin{equation} \label{eq:vertical_oblique_transmission}
P(\phi'=A \mid \phi) = \begin{cases}
v + (1-v)p, & \text{if } \phi=A \\
(1-v)p, & \text{if } \phi=B
\end{cases},
\end{equation}
where $p=P(\phi=A)$ is the frequency of $A$ among all adults in the parental generation.  

Not all adults become parents due to natural selection, and we denote the frequency of phenotype $A$ among parents with $\tilde{p}$.
Therefore, the frequency $\hat{p}$ of  phenotype $A$ among juveniles (after selection and vertical and oblique transmission) is

\begin{equation}\label{eq:horizontal}
\begin{aligned}
\hat{p}
& = \tilde{p} [v + (1-v)p] + (1-\tilde{p}) [(1-v)p] \\
& = v \tilde{p} + (1-v) p.
\end{aligned}
\end{equation}

Individuals interact according to a prisoner's dilemma.
Specifically, individuals interact in pairs; a cooperator suffers a fitness cost $0<c<1$, and its partner gains a fitness benefit $b$, where we assume $b>c>0$. \textbf{\autoref{table:prisoner_payoff}} shows the payoff matrix, i.e. the fitness of an individual with phenotype $\phi_1$ when interacting with a partner of phenotype $\phi_2$.

%%% Table: payoff matrix
\begin{table}[h]
\centering
\begin{tabular}{lll}
\toprule
           & $\phi_2=A$ & $\phi_2=B$ \\ \cmidrule(r){1-3}
$\phi_1=A$ & $1+b-c$ & $1-c$ \\
$\phi_1=B$ & $1+b$   & $1$
\\ \bottomrule
\end{tabular}
\caption{\textbf{Payoff matrix for prisoner's dilemma.}
The fitness of phenotype $\phi_1$ when interacting with phenotype $\phi_2$. $A$ is a cooperative phenotype, $B$ is a defector phenotype, $b$ is the benefit gained by an individual interacting with a cooperator, and $c$ is the cost of cooperation. $b>c>0$.
}
\label{table:prisoner_payoff}
\end{table}
%%%

Social interactions occur randomly:
two individuals with phenotype $A$ interact with probability $\hat{p}^2$, two individuals with phenotype $B$ interact with probability $(1-\hat{p})^2$, and two individuals with different phenotypes interact with probability $2\hat{p}(1-\hat{p})$. 

Horizontal cultural transmission occurs between pairs of individuals from the same generation. 
It occurs between social partners with probability $\alpha$, or between a random pair with probability $1-\alpha$ (see~\textbf{\autoref{fig:horizontal}}).
The assortment parameter $\alpha$ is therefore the fraction of population that receives (horizontal transmission) from the social interaction partner, and $1-\alpha$ receives randomly.
Horizontal transmission is not always successful, as one partner may reject the other's phenotype. The probability for successful horizontal transmission of phenotypes $A$ and $B$ are $T_A$ and $T_B$, respectively (\textbf{\autoref{table:interactions}}).

Therefore, the frequency $p'$ of phenotype $A$ among adults in the next generation, after horizontal transmission, is 
\begin{equation}\label{eq:nextgen_adults}
\begin{aligned}
p'
& = \hat{p}^2 [\alpha + (1-\alpha)(\hat{p} + (1-\hat{p})(1-T_B))] \\
& + \hat{p}(1-\hat{p}) [\alpha(1-T_B) + (1-\alpha)(\hat{p} + (1-\hat{p})(1-T_B))] \\
& + (1-\hat{p})\hat{p} [\alpha T_A + (1-\alpha) \hat{p} T_A ] \\
& + (1-\hat{p})^2 [(1-\alpha) \hat{p} T_A] \;,
\end{aligned}
\end{equation}
which simplifies to
\begin{equation}\label{eq:nextgen_adults_slimpify}
p' = \hat{p}^2(T_B-T_A) + \hat{p}(1+T_A-T_B) .
\end{equation}

The frequency of $A$ among parents (i.e. after selection) follows a similar dynamic, but also includes the effect of natural selection, and is therefore
\begin{equation}\label{eq:nextgen_parents}
\begin{aligned}
\bar{w} \tilde{p}'
& = \hat{p}^2 (1+b-c) [\alpha + (1-\alpha)(\hat{p} + (1-\hat{p})(1-T_B))] \\
& + \hat{p}(1-\hat{p}) (1-c) [\alpha(1-T_B) + (1-\alpha)(\hat{p} + (1-\hat{p})(1-T_B))] \\
& + (1-\hat{p})\hat{p} (1+b) [\alpha T_A + (1-\alpha) \hat{p} T_A ] \\
& + (1-\hat{p})^2 [(1-\alpha) \hat{p} T_A] \;,
\end{aligned}
\end{equation}
where fitness values are taken from \textbf{\autoref{table:prisoner_payoff}} and \textbf{\autoref{table:interactions}}, and the population mean fitness is
\begin{equation} \label{eq:mean_fitness}
\bar{w} =  1 + \hat{p}(b-c).
\end{equation}

\autoref{eq:nextgen_parents} can be simplified to
\begin{equation}\label{eq:nextgen_parents_simplified}
\begin{aligned}
\bar{w} \tilde{p}'
& = \hat{p}^2 (1+b-c) \big(1-(1-\hat{p})(1-\alpha)T_B)\big) \\
& + \hat{p}(1-\hat{p}) (1-c) \big(\hat{p}(1-\alpha)T_B+1-T_B\big) \\
& + (1-\hat{p})\hat{p} (1+b) \big(\hat{p}(1-\alpha) + \alpha\big) T_A \\
& + (1-\hat{p})^2 \hat{p} (1-\alpha) T_A \;.
\end{aligned}
\end{equation}

%%% Table: interactions
\begin{table}[]
\begin{tabular}{@{}llllll@{}}
\toprule
\multirow{2}{*}{Phenotype $\phi_1$} &
  \multirow{2}{*}{Phenotype $\phi_2$} &
  \multirow{2}{*}{Frequency} &
  \multirow{2}{*}{Fitness of $\phi_1$} &
  \multicolumn{2}{l}{$P(\phi_1=A)$ via horizontal transmission:} \\ \cmidrule(l){5-6} 
    &     &                      &         & from partner, $\alpha$ & from population, $(1-\alpha)$ \\ \cmidrule(r){1-6}
$A$ & $A$ & $\hat{p}^2$          & $1+b-c$ & 1                      & $\hat{p}+(1-\hat{p})(1-T_B)$  \\
$A$ & $B$ & $\hat{p}(1-\hat{p})$ & $1-c$   & $1-T_B$                & $\hat{p}+(1-\hat{p})(1-T_B)$  \\
$B$ & $A$ & $\hat{p}(1-\hat{p})$ & $1+b$   & $T_A$                  & $\hat{p} T_A$                 \\
$B$ & $B$ & $(1-\hat{p})^2$      & $1$     & $0$                    & $\hat{p} T_A$                 \\ \bottomrule
\end{tabular}
\caption{\textbf{Interaction frequency, fitness, and transmission probabilities.}}
\label{table:interactions}
\end{table}
%%%

%%% Figure: model illustration
\begin{figure}[thb]
  \centering
  \includegraphics[scale=1]{figure1.pdf}
  \caption{\textbf{Cultural horizontal transmission.} Transmission occurs between interacting partners with probability $\alpha$ (left) or between two random peers with probability $1-\alpha$.}
  \label{fig:horizontal}
\end{figure}
%%%


%%%%%%%%%%%%%%%%%%%%%%%%%%%
%%% Results
\section*{Results}


%%%%%%%%%%%%%%%%%%%%%%%%%%%%%%%%%%%%%%%%%%%%%%%%
\subsection*{Oblique and Horizontal Transmission}

With only oblique and horizontal transmission, i.e. $v = 0$, \autoref{eq:horizontal} becomes $\hat{p}=p$ and \autoref{eq:nextgen_adults_slimpify} becomes % TODO check this, I changed the ref from nextgen_parents_simplified to nextgen_adults_slimpify
\begin{equation}  \label{eq:nextgen_parents_oblique_only}
p' = p^2 (T_B-T_A) + p (1+T_A-T_B) ,
\end{equation}

which gives the following result.\\

\begin{result}[Oblique and horizontal transmission of cooperation]
Without vertical transmission ($v=0$), if there is a horizontal transmission bias in favor of cooperation, namely
\begin{equation} \label{eq:oblique_only_result}
T_A > T_B, 
\end{equation}
then $p'>p$, and the frequency of the cooperator phenotype among adults increases every generation.
\end{result}

Therefore, in the absence of vertical transmission, selection plays no role in the evolution of cooperation. Hence, cooperation will evolve if the cooperator phenotype has a horizontal transmission bias (see~{\autoref{fig:results_c}).



%%%%%%%%%%%%%%%%%%%%%%%%%%%%%%%%%%%%%%%%%%
\subsection*{Vertical and Horizontal Transmission}

With only vertical and horizontal transmission, i.e. $v=1$, \autoref{eq:horizontal} becomes
$\hat{p} =  \tilde{p}$,
and \autoref{eq:nextgen_parents_simplified} for the frequency of the cooperative phenotype among parents in the next generation $\tilde{p}'$ can be written as
\begin{equation} \label{eq:nextgen_parents_vertical_only} 
\begin{aligned}
\bar{w} \tilde{p}' 
& = \tilde{p}^2 (1+b-c) [1 - (1-\tilde{p}) (1-\alpha) T_B] \\
& + \tilde{p}(1-\tilde{p}) (1-c) [\tilde{p} (1-\alpha) T_B + 1 - T_B] \\
& + \tilde{p}(1-\tilde{p}) (1+b) [\tilde{p} (1-\alpha) + \alpha] T_A \\
& + (1-\tilde{p})^2 \tilde{p} (1-\alpha) T_A .
\end{aligned}
\end{equation}

The fixation of either cooperation or defection, 
$\tilde{p}=0$ and $ \tilde{p}=1$, are equilibria of \autoref{eq:nextgen_parents_vertical_only}, that is, they solve $\tilde{p}'= \tilde{p}$.
We therefore assume for the remainder of the analysis that $0<\tilde{p}<1$.

If $\alpha=1$, then $\tilde{p}'= \tilde{p}$ is reduced to
\begin{equation}
\tilde{p}(1-\tilde{p})\big[(1+b)T_A + (1-c)(1-T_B)-1\big] = 0,
\end{equation}
and there are no additional equilibria.

Therefore, for cooperation to take over the population (for $\tilde{p}=1$ to be globally stable) we require $\tilde{p}'>\tilde{p}$, that is,
\begin{equation}
  \tilde{p}^2 (1+b-c) + \tilde{p}(1-\tilde{p}) \big[(1-c) (1 - T_B) + (1+b)T_A\big] 
  > \bar{w}\tilde{p} .
\end{equation}
We divide by $\tilde{p}$, set $\bar{w} = 1 + \tilde{p}(b-c)$, and rearrange to get
\begin{equation}
  (1-\tilde{p}) \big[(1-c) (1 - T_B) + (1+b)T_A\big] 
  > 1 -\tilde{p} .
\end{equation}
Dividing by $(1-\tilde{p})$ we find that $\tilde{p}'>\tilde{p}$ if 
\begin{equation} \label{eq:vert_hori_alpha1_condition_proof}
  (1-c) (1 - T_B) + (1+b)T_A
  > 1
\end{equation}
\\

If $\alpha<1$, we want to determine a condition for $\tilde{p}'>\tilde{p}$. 
We divide \autoref{eq:nextgen_parents_vertical_only} by $\tilde{p}$ and set $\bar{w} = 1 + \tilde{p}(b-c)$ to get
\begin{equation}
\begin{aligned} 
  1 + \tilde{p}(b-c) < 
  & \, \tilde{p}(1+b-c) (1 - (1-\tilde{p}) (1-\alpha) T_B) \\
  & + (1-\tilde{p}) (1-c) (\tilde{p} (1-\alpha) T_B + 1 - T_B) \\
  & + (1-\tilde{p}) (1+b) (\tilde{p} (1-\alpha) + \alpha) T_A \\
  & + (1-\tilde{p})^2 (1-\alpha) T_A .
\end{aligned}
\end{equation}
Rearranging, we get
\begin{equation} 
\begin{aligned} 
  1 - \tilde{p} < 
  & - \tilde{p}(1+b-c)(1-\tilde{p}) (1-\alpha) T_B \\
  & + (1-\tilde{p}) (1-c) (\tilde{p} (1-\alpha) T_B + 1 - T_B) \\
  & + (1-\tilde{p}) (1+b) (\tilde{p} (1-\alpha) + \alpha) T_A \\
  & + (1-\tilde{p})^2 (1-\alpha) T_A .
\end{aligned}
\end{equation}
Diving by $(1-\tilde{p})$ and rearranging so that free terms are on the left and terms with $\tilde{p}$ are on the right, we have
\begin{equation} 
\begin{aligned} 
  &1 - (1-\alpha) T_A - (1+b) \alpha T_A - (1 - T_B)(1-c)  < \\
   &\tilde{p}[ - (1+b-c) (1-\alpha) T_B 
   + (1-c) (1-\alpha) T_B
   + (1+b) (1-\alpha) T_A 
   - (1-\alpha) T_A].
\end{aligned}
\end{equation}
Simplifying, we find that $\tilde{p}'>\tilde{p}$ if and only if
\begin{equation} \label{eq:vert_hori_global_condition}
c(1-T_B) - b \alpha T_A - (T_A-T_B) < \tilde{p} \cdot b (1-\alpha) (T_A-T_B).
\end{equation}

Following the same steps to solve $\tilde{p}'=\tilde{p}$, we find that there can be a third, polymorphic equilibrium 
\begin{equation} \label{eq:vert_hori_equilibrium}
  \tilde{p}^* = 
  \frac{c(1-T_B) - b \alpha T_A - (T_A-T_B)}{b (1-\alpha) (T_A-T_B)} .
\end{equation} 
Note that this is a legitimate equilibrium only if $0<\tilde{p}^*<1$.

Note that all parameters are positive.
So, applying \autoref{eq:vert_hori_global_condition}, for $\tilde{p}'>\tilde{p}$ we require that either 
\begin{align} 
\label{eq:vert_hori_global_condition1}
T_A > T_B \quad&\text{and}\quad \tilde{p}>\tilde{p}^*,  \quad \text{or} \\
\label{eq:vert_hori_global_condition2}
T_A <T_B \quad&\text{and}\quad \tilde{p}<\tilde{p}^* .
\end{align}

We therefore have the following result and corollaries.\\

\begin{result}[Vertical and horizontal transmission of cooperation]
Without oblique transmission ($v=1$), fixation, extinction, and coexistence of both phenotypes are possible.
\end{result}

We define the initial frequency as $\tilde{p}_0$ and the cost boundaries
\begin{equation}\begin{aligned}
\gamma_1 = \frac{b \alpha T_A - (T_A - T_B)}{1-T_B}, \quad
\gamma_2 = \frac{b \alpha T_B + (1+b) (T_A - T_B)}{1-T_B}.
\end{aligned}\end{equation}
Applying eqs.~\ref{eq:vert_hori_equilibrium}, \ref{eq:vert_hori_global_condition1}, and \ref{eq:vert_hori_global_condition2} we can summarize the possible outcomes:
%\begin{enumerate} % p* version
%\item Fixation of cooperation,
%	if $T_A>T_B$ and $0<\tilde{p}^*$, or
%	if $T_A>T_B$ and $0<\tilde{p}^*<1$, and $\tilde{p}_0>\tilde{p}^*$, or
%	if $T_A<T_B$ and $1<\tilde{p}^*$.
%\item Global fixation of defection,
%	if $T_A>T_B$ and $\tilde{p}^*>1$, or
%	if $T_A>T_B$ and $0<\tilde{p}^*<1$, and $\tilde{p}_0<\tilde{p}^*$, or
%	if $T_A<T_B$ and $\tilde{p}^*<0$
%\item Co-existence of both phenotypes at $\tilde{p}^*$, 
%	if $T_A<T_B$ and $0<\tilde{p}^*<1$.
%\end{enumerate}
\begin{enumerate} % gamma version
\item Fixation of cooperation,
	if $T_A>T_B$ and $c<\gamma_1$; or
	if $T_A>T_B$ and $\gamma_1<c<\gamma_2$ and $\tilde{p}_0>\tilde{p}^*$; or
	if $T_A<T_B$ and $c<\gamma_2$.
\item Fixation of defection,
	if $T_A>T_B$ and $\gamma_2<c$; or
	if $T_A>T_B$ and $\gamma_1<c<\gamma_2$ and $\tilde{p}_0<\tilde{p}^*$; or
	if $T_A<T_B$ and $\gamma_1<c$.
\item Coexistence of both phenotypes at $\tilde{p}^*$, 
	if $T_A<T_B$ and $\gamma_2<c<\gamma_1$.

Much of the literature on evolution of cooperation focuses on conditions for cooperation to invade a population of defectors.
The next corollary deals with such a condition.
\\
\end{enumerate}

\begin{corollary}[Condition for cooperation to increase from rarity]
If the initial frequency of the cooperative phenotype if very close to zero, $\tilde{p}_0 \approx 0$, then its frequency will increase if 
\begin{equation} \label{eq:unequal_transmission_from_rarity}
\begin{aligned}
T_A>T_B \;\; \text{and} \;\; c < \gamma_1, \quad \text{or} \quad
T_A<T_B \;\; \text{and} \;\; \gamma_2<c < \gamma_1. 
\end{aligned}
\end{equation} 
\end{corollary}

In general, these conditions cannot be formulated in the form of Hamilton's rule ($c<b\cdot r$) due to the horizontal transmission bias $T_A-T_B$.
Without horizontal transmission bias, we get the following corollary that does have the form of Hamilton's rule.\\

\begin{corollary}[Symmetric horizontal transmission]
If $T=T_A=T_B$, then cooperation will take over the population if
\begin{equation}
\label{eq:equal_transmission}
c < b \cdot \frac{\alpha T}{1-T}.
\end{equation}
\end{corollary}
To verify, set $T_A=T_B$ in \autoref{eq:vert_hori_global_condition}.

This can be interpreted as a version of Hamilton's rule (\autoref{eq:hamilton_rule}), where $\alpha T/(1-T)$ is the `effective relatedness'.
\autoref{fig:results_a} demonstrates this condition. 
\\

\begin{corollary}[No assortment of transmission and cooperation]
When $\alpha=0$, then either
\textbf{(1)} cooperation can fix, but not increase from rarity, if there is horizontal bias for cooperation ($T_A>T_B$) and the cost is low enough ($c<(1+b)(T_A-T_B)/(1-T_B)$), or 
\textbf{(2)} cooperation can increase from rarity to a stable coexistence with defection, if there is horizontal bias for defection ($T_A<T_B$) and the cost is low enough ($c<(T_B-T_A)/(1-T_B)$).
\end{corollary}

Here, the third equilibrium is
\begin{equation} \label{eq:vert_hori_alpha0_equilibrium}
\tilde{p}^*(\alpha=0) = \frac{c(1-T_B) - (T_A-T_B)}{b (T_A-T_B) },
\end{equation} 
and the cost boundaries are
\begin{equation}\begin{aligned}
\gamma_1(\alpha=0) = \frac{T_B - T_A}{1-T_B}, \quad
\gamma_2(\alpha=0) = \frac{(1+b) (T_A - T_B)}{1-T_B}.
\end{aligned}\end{equation}
If $T_A>T_B$ then $\gamma_1(\alpha=0)<0<\gamma_2(\alpha=0)$.
So cooperation cannot increase from rarity, but it can fix if $c<\gamma_2(\alpha=0)$.
If $T_A<T_B$ then $\gamma_2(\alpha=0)<0<\gamma_1(\alpha=0)$.
So cooperation cannot fix, but if $c<\gamma_1(\alpha=0)$ then it can increase from rarity to a stable coexistence at $\tilde{p}^*(\alpha=0)$.
\\

\begin{corollary}[Complete assortment of transmission and cooperation]
When $\alpha=1$, there are only two equilibria, $\tilde{p}=0$ and $\tilde{p}=1$.
The condition for evolution of cooperation (i.e. global stability of $\tilde{p}=1$) is found by setting $\tilde{p}'>\tilde{p}$, which gives
\begin{equation}\label{eq:vert_hori_alpha1}
c < \frac{b \cdot T_A + (T_A - T_B)}{1-T_B}.
\end{equation}
\end{corollary}
This is proven in \autoref{eq:vert_hori_alpha1_condition_proof}.
In this case there is complete assortment, and horizontal transmission always occurs together with the cooperative interaction. The same occurs in \citet{lewin2017microbes}, and therefore this corollary is equivalent to their result, see their eq.~1.

In terms of the cost boundaries, \autoref{eq:vert_hori_alpha1} is equivalent to $c<\gamma_1$. If $T_A>T_B$ then that suffices for fixation of cooperation. If $T_B>T_A$ then $\gamma_2(\alpha=1)<0$ and again, \autoref{eq:vert_hori_alpha1} is sufficient for increase in frequency of $A$ up to $\tilde{p}^*(\alpha=1) \approx \infty$.

\autoref{eq:vert_hori_alpha1} can be written as
\begin{equation} \label{eq:vert_hori_alpha1_effective}
1 - (1-c)(1-T_B) < (1+b) T_A ,
\end{equation}
which provides an interesting interpretation for the success of cooperation. 
Consider an interaction between two individuals: a cooperator and a defector.
$(1-c)(1-T_B)$ is the probability that the cooperator remains cooperative and also reproduces. 
Therefore, $1 - (1-c)(1-T_B)$ is the probability that either the cooperator becomes a defector, \emph{or} that it fails to reproduce.
This is the effective cost for cooperation from this interaction.
$(1+b) T_A$ is the probability that the defector becomes cooperative and reproduces.
This is the effective benefit for cooperation from this interaction.
So, \autoref{eq:vert_hori_alpha1} means that cooperation can evolve if the effective cost for cooperation is less than the effective benefit.



%%%%%%%%%%%%%%%%%%%%%%%%%%%%%%%%%%%%%%%%%%%%%%%%
\subsection*{With Vertical and Oblique Transmission}

In this case $0<v<1$, and the recursion system is more complex.
Therefore, we focus on local stability, rather than global stability.
To proceed, we note that 
\autoref{eq:horizontal} can give $\hat{p}'$ as a function of both $p'$ and $\tilde{p}'$,
\autoref{eq:nextgen_adults_slimpify} gives $p'$ as a function of $\tilde{p}$, and 
\autoref{eq:nextgen_parents_simplified} gives $\tilde{p}'$ as a function of $\hat{p}$. 
Combining these equations, we find an equation for $\hat{p}'$ as a function of $\hat{p}$, see Appendix~\autoref{sec:appendixB}.
We then determine the equilibria, which are solutions of $\hat{p}' = \hat{p}$, and analyse their local stability.
%: an equilibrium $\hat{p}^*$ is locally stable when the derivative of $f(\hat{p})=\bar{w}(\hat{p}'-\hat{p})$ at the equilibrium is negative, $f'(\hat{p}^*)<0$. 

%We start with the simple case of symmetrical horizontal transmission, $T=T_A=T_B$ and apply \autoref{eq:horizontal}, \autoref{eq:nextgen_adults_slimpify}, and \autoref{eq:nextgen_parents_simplified} to obtain 
%\begin{equation} \label{eq:equal_horizontal_transmission}
%\begin{aligned}
%  f(\hat{p}) &= 
%  \bar{w}(\hat{p}' - \hat{p}) = \\
%  &\hat{p}(1-\hat{p})\big[\alpha bvT - cv(1-T)\big].\end{aligned}
%\end{equation}
%A detailed explain of how we get eq.~\ref{eq:equal_horizontal_transmission} can be found in appendix B.
%The equilibria are solutions of $f(\hat{p})=0$, or $\hat{p}' = \hat{p}$.
%It is easy to verify that fixation of either phenotype, $\hat{p} =  0$ and $\hat{p} = 1$, is an equilibrium.
%Since the derivative of $f'(\hat{p}$ is 
%\begin{equation}
%f'(\hat{p})=(1-2\hat{p})\big[\alpha bvT - cv(1-T)\big],
%\end{equation}
%then the condition for local stability of $\hat{p}=1$ is
%\begin{equation} \label{eq:derivative_of_phattag-phat}
%	f'(1) =	-\alpha bvT + cv(1-T) < 0,
%\end{equation}
%which gives the following result.
%
%\textbf{Result 3: Oblique and vertical transmission with symmetric horizontal transmission.}
%If horizontal transmission is symmetric, $T = T_A = T_B$, and if
%\begin{equation} \label{eq:oblique_and_vertic_result2}
%  %\frac{b}{c}>\frac{1-T}{\alpha T}
%	c < b \cdot \frac{\alpha T}{1-T},
%\end{equation}
%then fixation of the cooperator phenotype $A$ is locally stable.
%The same condition was given in Corollary 1.1, \autoref{eq:equal_transmission}.

%We now turn to the general case where $T_A \neq T_B$. 
We apply \autoref{eq:horizontal}, \autoref{eq:nextgen_adults_slimpify}, and \autoref{eq:nextgen_parents_simplified} to obtain the function $f(\hat{p})$, see Appendix~\autoref{sec:appendixB}:

\begin{equation} \label{eq:general_case_polynomial}
  f(\hat{p}) = \bar{w}(\hat{p}'-\hat{p}) =
  \beta_1 \hat{p}^3 + \beta_2 \hat{p}^2 + \beta_3 \hat{p},
\end{equation}
where 
\begin{equation} \label{eq:polynomial_coefficients}
\begin{aligned}
\beta_1 &= \big[c(1-v) - b (1-\alpha v)\big] (T_A-T_B) , \\
\beta_2 &= -\beta_1 -\beta_3 ,  \\
\beta_3 &= \alpha bvT_A - cv(1-T_B) + (T_A-T_B) .
\end{aligned}
\end{equation}

If $T=T_A=T_B$ then $\beta_1=0$ and $\beta_3=-\beta_2=\alpha b vT -cv(1-T)$. 
Therefore, $f(\hat{p})$ is a quadratic polynomial,
\begin{equation} \label{eq:equal_horizontal_transmission}
  f(\hat{p}) = \hat{p}(1-\hat{p})\big[\alpha bvT - cv(1-T)\big].
\end{equation}
Clearly the only two equilibria are the fixations of either phenotype, $\hat{p} =  0$ and $\hat{p} = 1$.
These equilibria are locally stable if $f'(\hat{p})<0$ (Appendix~\autoref{sec:appendixC}).
Therefore, we find the derivative,
\begin{equation}
f'(\hat{p})=(1-2\hat{p})\big[\alpha bvT - cv(1-T)\big],
\end{equation}
and investigate its sign at the equilibria,
\begin{equation} \label{eq:derivative_of_phattag-phat}
\begin{aligned}
	f'(0) &=	\alpha bvT - cv(1-T), \\
	f'(1) &=	-\alpha bvT + cv(1-T).
\end{aligned}
\end{equation}
Therefore with symmetric horizontal transmission, fixation of the cooperative phenotype ($\hat{p}=1$) occurs under the same condition as Corollary 1.1, \autoref{eq:equal_transmission}.


In the general case where $T_A \neq T_B$, the coefficient $\beta_1$ is not necessarily zero, and $f(\hat{p})$ is a cubic polynomial.
Therefore, three equilibria may exist, two of which are
$\hat{p} = 0 $ and $\hat{p} = 1$.
By solving $f(\hat{p})/\big[\hat{p}(1-\hat{p})\big] = \beta_3 -\beta_1 \hat{p} = 0$ we  find the third equilibrium
\begin{equation} \label{eq:oblique_and_vertic_result}
  \hat{p}^* =  
  \frac{\beta_3}{\beta_1}.
\end{equation}

Note that the sign of this cubic at positive (negative) infinity is equal (opposite) to the sign of $\beta_1$. 
If $T_A>T_B$, then 
\begin{equation} \label{eq:beta1}
   \beta_1 < [c(1-\alpha v) - b(1-\alpha v)] (T_A-T_B) 
   = (1-\alpha v)(c-b)(T_A-T_B) < 0 ,
 \end{equation}
since $c<b$ and $1>\alpha v$, the sign of the cubic at positive and negative infinity is negative and positive, respectively.
First, if $\beta_3<\beta_1$ then 
$1<\hat{p}^*$ and therefore $f'(0)<0$ and $f'(1)>0$, that is, fixation of the defector phenotype $B$ is the only locally stable legitimate (i.e. between 0 and 1) equilibrium.
Second, if $\beta_1<\beta_3<0$ then 
$0<\hat{p}^*<1$ and therefore $f'(0)<0$ and $f'(1)<0$, that is, both fixations are locally stable and $\hat{p}^*$ separates the domains of attraction.
Third, if $0<\beta_3$ then 
$\hat{p}^*<0$ and therefore $f'(0)>0$ and $f'(1)<0$, that is, fixation of the cooperator phenotype $A$ is the only locally stable legitimate equilibrium.

Similarly, if $T_B>T_A$, then
\begin{equation} \label{eq:beta1_rev}
   \beta_1 > [c(1-\alpha v) - b(1-\alpha v)] (T_A-T_B) 
   = (1-\alpha v)(c-b)(T_A-T_B) > 0,
 \end{equation}
since $c<b$, and $1>\alpha v$. So the sign of the cubic at positive and negative infinity is positive and negative, respectively. 
First, if $\beta_3<0$ then $\hat{p}^*<0$ and therefore $f'(0)<0$ and $f'(1)>0$, that is, fixation of the defector phenotype $A=B$ is the only locally stable legitimate equilibrium.
Second, if $0<\beta_3<\beta_1$ then $0<\hat{p}^*<1$ and therefore $f'(0)>0$ and $f'(1)>0$, that is, both fixations are locally unstable and $\hat{p}^*$ is a stable polymorphic equilibrium.
Third, if $\beta_1<\beta_3$ then $\hat{p}^*>1$ and therefore $f'(0)>0$ and $f'(1)<0$, that is, fixation of the cooperator phenotype $A$ is the only locally stable legitimate equilibrium.

The following result summarizes these findings.

\begin{result}[Vertical, oblique, and horizontal transmission of cooperation]
The cultural evolution of a cooperator phenotype will follow one of the following scenarios, depending on the horizontal transmission bias $T_A-T_B$ and the coefficients $\beta_1$ and $\beta_3$:
\begin{enumerate}
\item \emph{Fixation of the cooperative phenotype $A$}, 
\begin{enumerate}
\item if $T=T_A=T_B$ and $c < b\cdot \frac{\alpha T}{1-T}$, or
\item if $T_A>T_B$ and $0<\beta_3$, or 
\item if $T_A<T_B$ and $\beta_1<\beta_3$.
\end{enumerate}

\item \emph{Fixation of the defector phenotype $B$}, 
\begin{enumerate}
\item if $T=T_A=T_B$ and $c > b\cdot \frac{\alpha T}{1-T}$, or 
\item if $T_A>T_B$ and $\beta_1<\beta_3<0$, or 
\item if $T_A<T_B$ and $\beta_3<0$.
\end{enumerate}

\item \emph{Protected polymorphism, or co-existence of both phenotypes}, if $T_A < T_B$ and $0<\beta_3<\beta_1$.

\item \emph{Fixation of either phenotype depending on initial frequency}, if $T_A>T_B$ and $\beta_3<\beta_1$.

\end{enumerate}
\end{result}



\begin{figure}[H]
  \centering
  \begin{subfigure}{8cm}
    \includegraphics[scale=0.5]{figure2a.pdf}
    \caption{$v=1$, $T_A=T_B=T$, $\alpha \neq 0$}
    \label{fig:results_a}
  \end{subfigure}
  \begin{subfigure}{8cm}
    \includegraphics[scale=0.5]{figure2b.pdf}
    \caption{$v=1$, $\alpha = 0$}
    \label{fig:results_b}
  \end{subfigure}
  \begin{subfigure}{8cm}
    \includegraphics[scale=0.5]{figure2c.pdf}
    \caption{$v=0$}
    \label{fig:results_c}
  \end{subfigure}
  \label{fig:results}
  \caption{
  \textbf{Numerical results for cultural evolution of cooperation.}
  Shown are dynamics of \textbf{(a-b)} $\tilde{p}$, the frequency of parents with cooperative phenotype $A$; \textbf{(c)} $p'$, the frequency of adults with cooperative phenotype $A$.
  The figure demonstrates fixation of cooperation (green), extinction of cooperation (blue)m and stable co-existence of cooperators and defectors (orange).
  %TODO ic (c), the blue and green got mixed. 
  % TODO have all panels in one row.
  }
\end{figure}


%%%%%%%%%%%%%%%%%%%%%%%%%%%
% Discussion
\section*{Discussion}
We hypothesized that non-vertical transmission can explain the evolution of cooperation.
We studied fully mixed and very large populations with a prisoner's dilemma payoff. 
We found that under horizontal and vertical cultural transmissions, if \autoref{eq:unequal_transmission} is satisfied, cooperation will take over fully mixed populations (Result 1).
Under oblique and horizontal transmission, horizontal transmission bias for the cooperative phenotype is sufficient and necessary for evolution of cooperation (Result 2, \autoref{eq:oblique_only_result}).
Under a combination of vertical, oblique, and horizontal transmission the dynamics are further complicated. Yet, we find that cooperation can evolve and in some cases be maintained together with defection (Result 4). 
Importantly, our results demonstrate that cooperation can evolve even in a fully mixed population (i.e. in an unstructured population), without repeating interactions or individual recognition.
These results significantly further our understating of the cultural evolution of cooperation. 

This study was partially inspired by \citet{lewin2017microbes}. 
They hypothesised that microbes that manipulate their hosts to act altruistically can be favored by selection, and may play a role in the widespread occurrence of cooperative behavior. Indeed, it has been shown that microbes can mediate behavioral changes in their hosts~\citep{dobson1988population,poulin2010parasite}. Therefore, natural selection on microbes may favor manipulation of the host so that it cooperates with others. Microbes can be transmitted \emph{horizontally} from one host to another during host interactions, and following horizontal transfer, the recipient host may carry microbes that are closely related to the microbes of the donor host, even when the two hosts are (genetically) unrelated~\citep{lewin2017microbes}. Microbes can also be transferred vertically, from parent to offspring, and % yr: cite 10.1126/science.aat7164
a microbe that induces its host to cooperate with another host and thereby increases the latter's fitness will  increase the vertical transmission of the microbes of the receiving individual. Kin selection among microbes could therefore favor microbes that induce cooperative behavior in their hosts, thereby increasing the transmission of their microbial kin.




There is an ongoing debate about the extent to which kin selection explains the evolution of cooperation and altruism.
For example, it has been suggested that it can explain the cooperative behavior of worker castes of eusocial insects like the honey bee. % TODO REF
The most significant argument against kin selection is that in some cases cooperation among unrelated individuals appears to have evolved~\citep{wilson2005kin}.
Therefore, other theories have been develop to explain the evolution of cooperation and altruism.

\emph{Reciprocity} entails that repeated interactions or individual recognition are key components of the evolution of cooperation. In \emph{direct reciprocity} there are repeated encounters between the same two individuals, and at every encounter each individual has a choice between cooperation and defection. Hence, it may eventually pay off to cooperate if it may cause your partner to cooperate in the future.
This game-theoretic framework, known as the \emph{repeated prisoner's dilemma}, 
%A well known strategy to play this game is called  'tit-for-tat'. This strategy always starts with a cooperation, then it does whatever the other player has done in the previous round: a cooperation for a cooperation, a defection for a defection. There a unlimited number of possible strategies to play this game. However, 
can only lead to the evolution of cooperation if the cost is less than the benefit $b$ times the probability of another encounter between the same two individuals, $w$, 
%TODO Referencess
\begin{equation} \label{eq:reciprocity}
c < b \cdot w.
\end{equation}

Direct reciprocity assumes that both players are in a position to cooperate, but it can not explain cooperation in asymmetric interactions such as human philanthropy. \\
\emph{Indirect reciprocity} has also been suggested to explain the evolution of cooperation.
\citet{nowak2006five} claims that direct reciprocity is like a barter economy based on the immediate exchange of goods, while indirect reciprocity resembles the invention of currency. 
%TODO: refs
The currency that ``fuels the engines'' of indirect reciprocity is \emph{reputation}. 
However, reciprocity assumes repeated interactions and therefore has difficulty in explaining the evolution of cooperation if  interactions are not repeated. 

\emph{Group selection} theory posits that cooperation is favored because it imparts an advantage to the whole group, if selection acts at the group level in addition to the individual level. A common model for group selection divides the population into groups in which there are cooperators that help  other group members and defectors that do not.  % TODO add reference
Individuals reproduce proportionally to their fitness, and offspring are added to the same group as their parents.
If a group reaches a certain size it can split to two groups, so groups that grow faster will split more often.
Groups with cooperators grow faster than groups without cooperators, and
cooperation can evolve in this model when the cost $c$ is less than the benefit $b$ times the ratio between the  the number of groups $m$ and the sum of $m$ and the maximum group size $n$,
\begin{equation} \label{eq:groupselection}
c < b \cdot \frac{m}{m+n} .
\end{equation}

Group selection has been criticized by biologists who advocate a gene-centered view of evolution. % TODO ref
It has also been criticized because for cooperation to take over the population it must have higher fitness than defection, while under group selection theory the fitness of cooperators at the individual level is lower than the fitness of defectors. Thus a trait with a lower fitness taking over the population is a contradiction %TODO ref.
\citet{eldakar2011eight} reject this argument, claiming that it is a tautology and does not qualify as an argument against group selection. The distinction between individual and group selection requires a comparison of fitness differentials within and between groups in a multi-group population, and when a trait  evolves by group selection, despite having lower fitness within a group, that group might have higher average fitness in competition with other groups, all things considered. % TODO cite Simpsons' paradox?

\citet{Eshel1982} have shown that with assortative meeting, i.e. probability $m$ that individuals interact with within their phenotypic group, cooperation can evolve if $m > c/b$.
In our Corollary 1.1 (\autoref{eq:equal_transmission}), cooperation evolves if $\alpha T / (1-T) > c/b$. So in our model $\alpha T/(1-T)$ is the effective relatedness, which is affected by $\alpha$, the correlation between transmission and interaction, and $T$, the horizontal transmission rate.



% Javarone MA, Atzeni AE, Galam S. Emergence of Cooperation in the Prisoner ’ s Dilemma Driven by Conformity. 2:155-163. doi:10.1007/978-3-319-16549-3
% Woodcock S. The significance of non-vertical transmission of phenotype for the evolution of altruism. Biol Philos. 2006;21:213-234. doi:10.1007/s10539-005-8241-1


%%%%%%%%%%%%%%%%%%%%%%%%%%%
\pagebreak
% Acknowledgements
{\small
\section*{Acknowledgements}
We thank Lilach Hadany and Ayelet Shavit for discussions and comments.
This work was supported in part by
the Israel Science Foundation 552/19 (YR),
and Minerva Stiftung Center for Lab Evolution (YR).
% TODO funding for other authors?
}


%%%%%%%%%%%%%%%%%%%%%%%%%%%
% Appendices
\begin{appendices}
\renewcommand{\theequation}{\thesection\arabic{equation}}
\counterwithin*{equation}{section}

%%%%%%%%%%%%%%%%%%%%%%%%%%%%%%%%%%%%%%%%%%%%%%%%%%%%%
\section{} \label{sec:appendixB}
We want to find the frequency of juveniles with phenotype $A$ in next generation $\hat{p}'$ as a function of frequency
of juveniles with phenotype $A$ in the current generation $\hat{p}$.
We start from eq.~\ref{eq:horizontal},
\begin{equation}\label{eq:appendix_b_1}
  \hat{p}' = v \tilde{p}' + (1-v) p'
  \end{equation}
Substituting $p'$ using eq.~\ref{eq:nextgen_adults_slimpify} and $\tilde{p}'$ using eq.~\ref{eq:nextgen_parents_simplified}, we have
\begin{equation}\label{eq:appendix_b_2}
  \begin{aligned}
  \hat{p}'  = & \frac{v}{\bar{w}}\left\{\hat{p}^2(1+b-c)\left[1-(1-\hat{p})(1-\alpha)T_B)\right]\right\} \\
  & + \frac{v}{\bar{w}}\left\{ \hat{p}(1-\hat{p})(1-c)\left[ \hat{p}(1-\alpha)T_B + 1 - T_B \right] \right\} \\
  & + \frac{v}{\bar{w}}\left\{ \hat{p}(1-\hat{p})(1+b)\left[\hat{p}(1-\alpha) + \alpha \right]T_A \right\} \\
  & + \frac{v}{\bar{w}}(1-\hat{p})^2\hat{p}(1-\alpha)T_A \\
  & + (1-v)\hat{p}^2(T_B-T_A) + (1-v)\hat{p}(1+T_A-T_B),
\end{aligned}
\end{equation}
where $\bar{w} = 1 + \hat{p}(b-c)$.

We define $f(\hat{p})$ to be
\begin{equation} \label{eq:appendix_b_3}
\begin{aligned}
      f(\hat{p}) &= \bar{w}(\hat{p}' - \hat{p})
\end{aligned}
\end{equation}
Using \emph{SymPy} \citep{Meurer2017}, a Python library for symbolic mathematics, we simplify \autoref{eq:appendix_b_3} to 
\autoref{eq:general_case_polynomial} and \autoref{eq:polynomial_coefficients}.

%%%%%%%%%%%%%%%%%%%%%%%%%%%%%%%%%%%%%%%%%%%%%%%%%%%%
\section{} \label{sec:appendixC}

We show that $f'(\hat{p}^*)<0$ is a sufficient condition for local stability an equilibrium $\hat{p}^*$.
We will write $f(\hat{p})$ as a Taylor approximation around the equilibrium $\hat{p}^*$. 
\begin{equation} \label{eq:appendix_b_taylor}
  f(\hat{p}) = \bar{w}(\hat{p}'-\hat{p}) = f(\hat{p}^*) + f'(\hat{p}^*)(\hat{p}-\hat{p}^*) + R_2(\hat{p})
\end{equation}
Where $R_2(\hat{p})$ is the remainder. Since $\hat{p}^*$ is an equilibrium $f(\hat{p}^*)=0$.
\begin{equation} \label{eq:appendix_b_taylor0}
  f(\hat{p}) = \bar{w}(\hat{p}'-\hat{p}) =  f'(\hat{p}^*)(\hat{p}-\hat{p}^*) + R_2(\hat{p})
\end{equation}
Neglecting the remainder(?) and we get: 
\begin{equation} \label{eq:appendix_b_taylor_without_remainder}
  f(\hat{p}) = \bar{w}(\hat{p}'-\hat{p}) = f'(\hat{p}^*)(\hat{p}-\hat{p}^*)
\end{equation}
And we get:
\begin{equation} \label{eq:appendix_b_derivative}
  \begin{aligned}
  f'(\hat{p}^*) &= \bar{w}\Big(\frac{\hat{p}'-\hat{p}}{\hat{p}-\hat{p}^*}\Big)\\
  & = \bar{w}\Big(\frac{\hat{p}'-\hat{p}^*+\hat{p}^*-\hat{p}}{\hat{p}-\hat{p}}\Big) 
  = \bar{w}\Big(\frac{\hat{p}'-\hat{p}^*}{\hat{p}-\hat{p}^*}-1\Big) .
  \end{aligned}
\end{equation}
In order for the equilibrium $\hat{p}^*$ to be stable we demand that if $\hat{p} > \hat{p}^*$ then $\hat{p}'-\hat{p}^* < \hat{p}-\hat{p}^*$
and if $\hat{p} < \hat{p}^*$ we demand that $-(\hat{p}'-\hat{p}^*) < -(\hat{p}-\hat{p}^*)$. 
In other words, $\frac{\hat{p}'-\hat{p}^*}{\hat{p}-\hat{p}^*}<1$ for every $\hat{p}$.
Therefore,
\begin{equation} \label{eq:appendix_b_condition}
  \begin{aligned}
  f'(\hat{p}^*) = \bar{w}\Big(\frac{\hat{p}'-\hat{p}^*}{\hat{p}-\hat{p}^*}-1\Big) < 0 .
  \end{aligned}
\end{equation}
Therefore, $f'(\hat{p}^*)$ is a sufficient condition for stability of $\hat{p}^*$.

\end{appendices}

%%%%%%%%%%%%%%%%%%%%%%%%%%%
% Bibliography
\bibliographystyle{plainnat}
\bibliography{bib}

\end{document}
